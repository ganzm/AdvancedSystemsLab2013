\documentclass[milestone1.tex]{subfiles} 
\begin{document}

\section{Performance Relevant Features}
\subsection{Overview}

During a brainstorming session, a broad spectrum of performance relevant features were extracted, see figure~\ref{fig:features_mindmap}. Then, the primary features (PF, orange) and the secondary features (SF, green) were chosen according to the group membmers domain specific knowledge and presumptions. They were also validated during discussions with other group members and the teaching assistant.

%\begin{landscape}

%\tikzset{concept/.append style={fill={none}}}
\tikzset{concept/.append style={circle}}
\tikzset{level 1 concept/.append style={font=\sf, level distance = 44mm}}
\tikzset{level 2 concept/.append style={font=\sf, level distance = 22mm}}
\tikzset{level 3 concept/.append style={font=\sf, level distance = 22mm}}
\tikzset{every node/.append style={scale=0.8}}
    
\begin{figure}
\center
\begin{tikzpicture}[grow cyclic, align=flush center]
  \path[mindmap,concept color=black,text=white]
    node[concept] {Performance Relevant Features}
    %[clockwise from=0]
    child[concept color=orange] {
      node[concept] {Setup}
      child {
      	node[concept] {\# Clients} 
      }
      child[concept color=green!50!black] {
      	node[concept] {Use Shed Load} 
      }
      child {
      	node[concept] {\# Brokers} 
      }
    }
    child[concept color=white!35!black] {
      node[concept] {System}
      child[concept color=green!50!black] { 
        node[concept] {Garbage Collection}
      }
      child { 
        node[concept] {Hardware}
        child { node[concept] {CPU} }
        child { node[concept] {RAM} }
        child { node[concept] {Network} }
      }
      child { 
        node[concept] {Database Settings}
      }
      child[concept color=green!50!black] { node[concept] {Indices}}
    }
    child[concept color=orange] {
      node[concept] {Load types}
      child[concept color=orange] { 
      	node[concept] {Send Message} 
      }
      child[concept color=green!50!black] {
      	node[concept] {Peek Message} 
      }
      child[concept color=orange] {
      	node[concept] {Dequeue Message} 
      }
    }
    child[concept color=white!35!black] { 
      node[concept] {Queue / Pool Sizes}
      child[concept color=orange] { 
      	node[concept] {Worker pool size} 
      }
      child[concept color=orange] {
      	node[concept] {Database connection pool size} 
      }
      child[concept color=green!50!black] {
      	node[concept] {Request Queue Lenght} 
      }
      child[concept color=green!50!black] {
      	node[concept] {Response Queue Lenght} 
      }
    }
    child[concept color=white!35!black] { 
      node[concept] {Logging} 
      child { 
      	node[concept] {Perfor-mance Logging} 
      }
      child {
      	node[concept] {Debug Logging} 
      }
    }
    child[concept color=white!35!black] {
      node[concept] {Data in Database}
      child { 
      	node[concept] {\# Registered Clients} 
      }
      child {
      	node[concept] {\# Queues}
      }
      child {
      	node[concept] {\# Messages} 
      }
      child[concept color=green!50!black] {
      	node[concept] {Message Size} 
      }
    };
\end{tikzpicture}
\caption{Performance relevant features mind map} \label{fig:features_mindmap}
\end{figure}
%\end{landscape}

\subsection{Primary features}

According to figure~\ref{fig:features_mindmap}, the primary features are as follows:

\begin{enumerate}
  \item \textbf{\# Clients} The amount of clients interacting with the system.
  \item \textbf{\# Brokers} The amount of brokers interacting with the system.
  \item \textbf{Database connection pool size} The amount of database connections per broker to the database.
  \item \textbf{Worker pool size} The amount of workers per broker.
  \item \textbf{Send Message} When many insertions occur (send a message).
  \item \textbf{Dequeue Message} When many deletions occur (read a message and remove it from the queue).
\end{enumerate}


\subsection{Secondary features}

According to figure~\ref{fig:features_mindmap}, the secondary features are as follows:

\begin{enumerate}
  \item \textbf{Peek Message} When many selects occur (read a message, but don't remove it from the queue). This is mostly covered by the dequeue message feature and therefore only seen as a secondary feature.
  \item \textbf{Message Size} The average size of the messages payloads.
  \item \textbf{Use Shed Load} Whether the fault tolerance pattern Shed Load \footnote{http://dl.acm.org/citation.cfm?id=SERIES12798.1557393} is used or not.
  \item \textbf{Garbage Collection} The time it takes for the garbage collection.
  \item \textbf{Indices} Whether indices are used or not.
  \item \textbf{Request Queue Length} The amount of messages on a broker which are buffered until the are processed.
    \item \textbf{Response Queue Length} The amount of messages on a broker which are buffered until the are sent back to the client.
\end{enumerate}

\end{document}
