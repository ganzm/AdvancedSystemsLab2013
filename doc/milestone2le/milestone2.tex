\documentclass[a4paper]{article}

\usepackage{numprint}
\usepackage{nameref}
\usepackage{float}
\usepackage{url}
\usepackage{graphicx}	% For figure environment
\usepackage{epstopdf}
\usepackage[center]{subfigure}
\usepackage{amssymb}	% For mathematical figures like \mathbb{R}
\usepackage{amsmath}
\usepackage{framed}
\usepackage{tikz}
\usetikzlibrary{mindmap,trees}
\usepackage{pdflscape}
\usepackage[a4paper]{geometry}
\usepackage{subfiles}
\usepackage{listings}
\usepackage{color}

\definecolor{dkgreen}{rgb}{0,0.6,0}
\definecolor{gray}{rgb}{0.5,0.5,0.5}
\usepackage{array}
\usepackage{booktabs}% http://ctan.org/pkg/booktabs
\usepackage{xparse}% http://ctan.org/pkg/xparse
% Rotation: \rot[<angle>][<width>]{<stuff>}
\NewDocumentCommand{\rot}{O{45} O{1em} m}{\makebox[#2][l]{\rotatebox{#1}{#3}}}%
\definecolor{mauve}{rgb}{0.58,0,0.82}

\lstset{frame=tb,
  language=Java,
  aboveskip=3mm,
  belowskip=3mm,
  showstringspaces=false,
  columns=flexible,
  basicstyle={\small\ttfamily},
  numbers=none,
  numberstyle=\tiny\color{gray},
  keywordstyle=\color{blue},
  commentstyle=\color{dkgreen},
  stringstyle=\color{mauve},
  breaklines=true,
  breakatwhitespace=true
  tabsize=3
}


\title{Advanced Systems Lab - Milestone II} 
\author{Lukas Elmer (elmerl@ethz.ch)} 
\date{\today} 


\begin{document}

\maketitle

\pagebreak

\tableofcontents

\pagebreak

\begin{abstract}

This document is the follow up document of Advanced Systems Lab - Milestone I and describes an analytical queueing model of the system which was built using the operational laws. Using this model, the performance characteristics of the model are derived and compared to the measurements of Milestone I. Furthermore it is analysed where the data matches the model and where it does not.

\end{abstract}

\pagebreak

%% ----------------------------------------------
% Section Messaging System
%% ----------------------------------------------
\section{Tasks}

* analytical queuing model
** for each component of the system
** and for the system as a whole.

* derive the performance characteristics that the model predicts
** compare them with the results obtained in milestone 2

* Explain where the model and data match and where they do not match
** sufficiently detailed explanations of characteristics behind the observed behaviour
*** code
*** system
*** hardware

* the modeling should include for each component
** the queuing model(s)
** parameters

* overall system should be modeled
** as queuing network
** the experimental results analyzed using the operational laws

* clearly indicate
** the behavior expected from the model through graphs
** plot them together with the measured behavior

* When evaluating the experimental data and the models
** clearly indicate the laws you are applying
** explain why you think they can be applied in the corresponding analysis



%% ----------------------------------------------
\section{Components}

In the analysis, the focus of the system is on the middleware. Additionally, the database is modelled as a queue, but no further internals of the database are considered.\\

Figure \ref{fig:middleware-threading} shows a overview of the systems components. The most important parts of the analytical models are:

\begin{itemize}
\item The middleware (multiple instances)
	\begin{itemize}
	\item NIO (network input and output, single queue)
	\item Request queue (when the requests are waiting for workers)
	\end{itemize}
\item The database (modelled as a single queue)
\end{itemize}

Because the clients wait for the current message to be answered before sending the next message, the system is modelled as a \textbf{closed system}. If messages fail to be processed by the middleware, the client implements a backoff time, so the system doesn't get overloaded.

\section{Performance Characteristics}

As described in the book (TODO) in section 30.1, a queuing system can be specified by six parameters. Therefore we use the Kendall notation in the form A/S/m/B/K/SD, where the letters correspond to the six parameters listed on page 507-509 in the book.

\subsection{A: Arrival Process}

Even tough the clients send in a certain deterministic interval, because of the network and the operating system there is a delay until the requests arrive in the queuing system. We assume that this delay is exponentially distributed. Thus, A is the Markovian M. (can be seen in figure TODO)

\subsection{S: Service Time Distribution}

The service time distribution also is assumed to be distributed exponentially. We also know that the service time is memoryless - i.e. it does not matter what request happened before the current request. This can be assumed because there is no caching implemented.

\subsection{m: Number of Servers}

The number of servers is the amount of middlewares running.

\subsection{B: System Capacity}

The system capacity is defined by those who are waiting for service and those who receive service.  This is dependent on how many requests a middleware can accept. Because there are always less clients connected to a middleware then connections to the NIO part of the middlware can be established, the system capacity will be the same as the population size (TODO: check!!!).

\subsection{K: Population Size}

The population size is defined by the amount of clients issuing requests to the middleware.

\subsection{SD: Service Discipline}

In general, the service discipline is First Come, First Served (FCFS). However, because there are limited CPU's on the middleware, and CPU's usually use Round Robin (RR), this may have to be considered in the analysis.



\end{document}
